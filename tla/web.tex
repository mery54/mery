\documentclass[ 12pt]{article}
%\documentclass[ 12pt]{book}

\usepackage{times,vmargin,url,html}
\setmarginsrb{3cm}{1cm}{2cm}{1cm}{1cm}{1cm}{1cm}{1cm}
%\setpapersize{A4}
\usepackage{dr-bibbib} %-sort-year}

\usepackage{color,time,array,time}
\pagecolor{yellow}
%\pagecolor{cyan}
\pagecolor{white}

\newcommand{\keywords}[1]{\textbf{Keywords:} #1} %\\ \clearpage %\tableofcontents\clearpage}
%

\usepackage{xcolor}
%\usepackage{b2latex}

%\IEEEoverridecommandlockouts
% The preceding line is only needed to identify funding in the first footnote. If that is unneeded, please comment it out.
\usepackage{cite,time}
\usepackage{amsmath,amssymb,amsfonts}
\usepackage{algorithmic}
\usepackage{graphicx}
\usepackage{textcomp}
\usepackage{xcolor,time,url,nbcode,version}
\usepackage{wrapfig}
%\usepackage{b2latex}
\usepackage{caption}
\usepackage{bm}
\usepackage{float}
\usepackage{enumitem}
\usepackage{listings}
\usepackage{bussproofs}
\usepackage{wrapfig,setspace}
%\usepackage{lstcoq}
\usepackage{xspace}
\usepackage{hyperref}

\def\BibTeX{{\rm B\kern-.05em{\sc i\kern-.025em b}\kern-.08em
    T\kern-.1667em\lower.7ex\hbox{E}\kern-.125emX}}
\pagestyle{plain}
\pagenumbering{arabic}
\input{prelude}
%ddd -aal\renewcommand{\baselinestretch}{0.90}
\newtheorem{myexample}{Example}[section]
\newtheorem{mytheorem}{Property}
%\newtheorem{definition}{Definition}
%\includeversion{check}
% \excludeversion{check}
% \excludeversion{long}
%%%%%%%%%%%%%%%%%%%%%%%%%%%%%%%%%%%%%%%%%%%%%%%%
%%%%%%%%%%%%%%%%%%%%%%%%%%%%%%%%%%%%%%%%%%%%%%%%
\newtheorem{exemple}{Example}
%\newtheorem{theorem}{Theorem}
%\newcommand{\white}[1]{\textcolor{white}{#1}}

%\newtheorem{comm}{Teaching Point}
%\newenvironment{comm}{\iffalse}{\fi}
%\excludeversion{comm}

\input eb2latex

\pagestyle{plain}
%%%%%%%%%%%%%%%%%%%%%%%%%%%%%%%%%%%%%%%%%%%%%%%%
%%%%%%%%%%%%%%%%%%%%%%%%%%%%%%%%%%%%%%%%%%%%%%%%
\title{Modelling with TLA$^+$}
\author{Dominique M\'ery\\
LORIA \& Telecom Nancy\\ Universit\'e de Lorraine\\
\url{https://members.loria.fr/Mery}\\ \url{dominique-dot-mery-at-loria-dot-fr}}

%\date{Last updated \now on \today}

\makeatletter
\renewcommand\contentsname{Summary of the links}
\renewcommand\tableofcontents{%
  \null\hfill\textbf{\Large\contentsname}\hfill\null\par
  \@mkboth{\MakeUppercase\contentsname}{\MakeUppercase\contentsname}%
  \@starttoc{toc}%
}
\makeatother


\begin{document}
\newcounter{ex}  \setcounter{ex}{1}
\maketitle



\section{TLA$^+$ Models without Proofs}
\label{sec:last-news}

\subsection{ Computing the sum of elements of a vector.}
\label{sec:comp-sum-elem}


\begin{itemize}
\item[] 
  \href{https://mery54.github.io/mery/tla/TLAPROOFVECTSUM.tla}{TLA Module 
    for the algorithm  computing the sum of elements of a vector.}
\item   \href{https://mery54.github.io/mery/tla/TLAPROOFVECTSUM.pdf}{Module 
    TLA for the algorithm  computing the sum of elements of a vector
    in pdf}
    
\end{itemize}



\subsection{Incrementing a positive value}
\label{sec:comp-sum-elem}


\begin{itemize}
\item[] 
  \href{https://mery54.github.io/mery/tla/TLAPROOFINC.tla}{TLA Module 
    for the algorithm \textit{ incrementing a positive value.}}
\item   \href{https://mery54.github.io/mery/tla/TLAPROOFINC.pdf}{Module 
    TLA for the algorithm  \textit{incrementing a positive value}
    in pdf}
    
\end{itemize}



\subsection{Maximum of two integer values}
\label{sec:comp-sum-elem}


\begin{itemize}
\item[] 
  \href{https://mery54.github.io/mery/tla/TLAPROOFMAX2.tla}{TLA Module 
    for the algorithm \textit{ maximum of two integer values}}
\item   \href{https://mery54.github.io/mery/tla/TLAPROOFMAX2.pdf}{Module 
    TLA for the algorithm  \textit{{maximum of two integer values}}
    in pdf}
    
\end{itemize}

\end{document}
